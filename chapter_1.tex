\chapter{Lineair en geheeltallig programmeren}

\section{Introductie}
Lineair programmeren houdt in om een probleem te formuleren in termen van een \emph{doelfunctie} en \emph{beperkingen} in functie van de \emph{beslissingsvariabelen}. Bij geheeltallig programmeren komt hier nog een beperking bij, namelijk dat alle beslissingsvariabelen een geheeltallige waarde moeten hebben.

\section{Lineair programmeren}
Bij lineair programmeren dient een doelfunctie gemaximaliseerd te worden.
Indien het een minimalisatieprobleem is, worden alle co\"efficienten $b_1, \cdots, c_n$ van teken veranderd.

\begin{equation}
    \max{\sum_{i=1}^{n}{c_{i}x_i}}
\end{equation}

De beperkingen worden ook vaak in standaardvorm gezet, waarbij de rechterleden $b_1, \cdots, b_n$ niet-negatief zijn.
Ook de beslissingsvariabelen zelf dienen niet-negatief te zijn; indien dit wel het geval is kunnen deze uitgedrukt worden als het verschil van twee niet niet-negatieve variabelen.

\begin{align*} 
    \sum_{i=1}^{n}{a_{1i}x_i} & \sim  b_1 \\ 
    \vdots \\
    \sum_{i=1}^{n}{a_{mi}x_i} & \sim  b_m
\end{align*}

De tilde $\sim$ in voorgaande vergelijking stelt een gelijkheid of ongelijkheid voor, maar in de standaardvorm wordt gekozen voor gelijkheden. 
Door extra \emph{slack}-variabelen te introduceren, kunnen ongelijkheden ook voorgesteld worden.

\subsection{Gevoeligheidsanalyse}
Zodra een optimale oplossing bekend is, kan het nuttig zijn om te weten over welk bereik variabelen kunnen veri\"eren voordat de basis niet meer optimaal is. 
Voor twee variabelen kan deze analyse grafisch gebeuren.

Ook kan de schaduwprijs bepaald worden, dit is de hoeveelheid toename (voor een maximalisatieprobleem) wanneer het rechterlid van een beperking met 1 toeneemt, voor een bepaalde beperking. Men kan dit berekenen door een beperking te wijzigen naar:

\begin{align*} 
    \sum_{i=1}^{n}{a_{1i}x_i} & \sim  b_1 \\ 
    \vdots \\
    \sum_{i=1}^{n}{a_{ki}x_i} & \sim  b_k + \Delta \\ 
    \vdots \\
    \sum_{i=1}^{n}{a_{mi}x_i} & \sim  b_m
\end{align*}
