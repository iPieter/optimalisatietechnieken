\chapter{Dynamisch programmeren}
Dynamisch programmeren is in essentie ver doorgedreven recursie, waarbij een probleem wordt opgesplitst in kleinere deelproblemen.
Deze problemen kunnen dan onafhankelijk opgelost worden.
Een voorbeeld is het knapzakprobleem, waarvan het algoritme bescheven staat in algoritme~\ref{algo:Knapzak}.



%%%%%%%%%%%%%%%%%%
% Indien dit gekopieerd wordt, zie: https://en.wikibooks.org/wiki/LaTeX/Algorithms
%%%%%%%%%%%%%%%%%%
\begin{algorithm}
    \caption{Pseudocode van een oplossing voor het knapzakprobleem.}
    \label{algo:Knapzak}
    \begin{algorithmic}
        \Require{$cost[M]$}
        \Require{$size[N]$}
        \Require{$val[N]$}
        \ForAll{$j$ in $1$ to $N$ inclusive} \Comment{Iterate objects}
            \ForAll{$i$ in $1$ to $M$ inclusive} \Comment{Iterate volumes}
                \If{$i \geq size[j]$}
                    \If{$cost[i] \leq cost[i - size[j]] + val[j]$}
                        \State{$cost[i] \gets cost[i - size[j]] + val[j]$}
                        \State{$best[i] \gets = j$}
                    \EndIf
                \EndIf
            \EndFor    
        \EndFor
    \end{algorithmic}
\end{algorithm}
