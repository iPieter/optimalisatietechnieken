\chapter{Heuristieken en metaheuristieken}

\section{Introductie}
Heuristieken zijn informele methoden die \emph{voldoende goede} oplossingen kunnen opleveren voor complexe problemen. 
Hierbij worden twee soorten onderscheiden: constructieve en perturbatieve heuristieken.
Constructieve heuristieken bouwen een oplossing op en de groep van perturbatieve heuristieken maken wijzigingen aan een al bestaande oplossing.

\section{Modelleringen}
Een modellering dient om een oplossing voor te stellen, waarbij deze aan de volgende voorwaarden moet voldoen:

\begin{itemize}
    \item Volledigheid: alle mogelijke oplossingen kunnen voorgesteld worden in de modellering
    \item Connectiviteit: er bestaat een (indirect) pad tussen 2 oplossingen
    \item Effici\"entie: snel een eenvoudig te manipuleren en evalueren
\end{itemize}

\paragraph{Binaire codering}
Deze codering houdt voor ieder object een eigenschap bij. 
Bijvoorbeeld in geval van het knapzakprobleem, waarbij 0 voorstelt dat het item niet in de knapzak zit en 1 wel.
%
\begin{table}[!h]
    \centering
    \begin{tabular}{|l|l|l|l|l|l|l|}
    \hline
    1 & 0 & 0 & 1 & 1 & 1 & 0 \\ \hline
    \end{tabular}
\end{table}
%
\paragraph{Geheeltallige codering}
Deze codering kan meerdere eigenschappen toekennen aan een enkel object dan de binaire codering, wat voor toekennings- en locatieproblemen nuttig is.

Deze codering kan ook gebruikt worden om permutatie- en volgordeproblemen te weerspiegelen.
Hierbij kunnen nummers uiteraard geen twee keer voorkomen, tenzij de probleemstelling dat toelaat.
%
\begin{table}[!h]
    \centering
    \begin{tabular}{|l|l|l|l|l|l|l|}
    \hline
    4 & 7 & 8 & 1 & 2 & 6 & 5 \\ \hline
    \end{tabular}
\end{table}
%

\paragraph{Floating-point voorstelling}
Voor continue optimalisatieproblemen kan een voorstelling met zwevende komma's gebruikt worden.
Toepassingen zijn bijvoorbeeld het bepalen van samenstellingen of het optimaliseren vna parameters.

%
\begin{table}[!h]
    \centering
    \begin{tabular}{|l|l|l|l|l|l|l|}
    \hline
    2.5 & 0.8 & 10.3 & 1.0 & 2.3 & 4.6 & 0.1 \\ \hline
    \end{tabular}
\end{table}
%
\paragraph{Niet-lineaire voorstelling}
Mocht gekozen worden voor een niet-lineaire voorstelling van de oplossing, is dit in veel gevallen een graaf- of boomstructuur. 

\subsection{Directe en indirecte voorstelling}
Bij de modellering kan er gekozen worden om met een \emph{directe} voorstelling te werken, waarbij alle informatie direct beschikbaar is. Indien dit niet het geval is, en er dus een decoder nodig is om de oplossing te interpreteren, wordt gesproken van een \emph{indirecte} voorstelling.

\section{Lokale zoekmethoden}
Lokale zoekmethoden werken op basis van een initi\"ele oplossing, waarbij getracht wordt een lokaal minimum te bereiken. 
Deze algoritmen bestaan dus altijd uit twee stappen:

%
\begin{enumerate}
    \item Stap 1: construeer een initi\"ele oplossing.
    \item Stap 2: vervang de huidige oplossing door een betere buur.
    \item Stap 3: stoppen.
\end{enumerate}
%

Het bepalen van een valide initi\"ele oplossing kan op verschillende manieren gebeuren. 
Zo kan er vertrokken worden van een al gekende oplossing. 
Een andere strategie is om met een eenvoudigere heuristiek of zelfs willekeurig een oplossing te genereren.

De tweede stap is het vervangen van een oplossing door een betere uit de \emph{neighbourhood}. 
De strategie om deze oplossing te kiezen, bepaalt de performantie en het eindelijke resultaat.

Tenslotte moet er ook een stop-voorwaarde voorzien worden. 
Een eenvoudige voorwaarde is om de uitvoertijd te beperken, wat als nadeel heeft dat de oplossing mogelijks niet lokaal optimaal is.
Dit kan beter---maar ook niet gegarandeerd---bereikt worden door een timer te starten na de laatste verbetering.
Door een vast aantal iteraties te gebruiken in plaats van een timer, is dit wel mogelijk; dat aantal is echter afhankelijk van de probleemgrootte. 
Daarnaast hebben beide strategi\"en als nadeel dat ze minder duidelijk zijn naar de gebruiker toe. 

\paragraph{Steepest descent}
Hierbij wordt uit een omgeving de beste verbetering gekozen. 
Deze strategie heeft als nadeel dat deze snel vast komt te zitten in een lokaal optimum. Een voordeel is wel dat de stop-voorwaarde duidelijk is: wanneer er geen betere buren meer zijn. Daarnaast convergeert deze heuristiek ook snel, wat wilt zeggen dat het lokale optimum snel bereikt wordt.

\paragraph{Hill climbing}
Hier wordt niet de beste, maar eerste verbetering uit een omgeving aanvaard, wat uitaard dezelfde nadelen heeft als steepest descent.

\section{Metaheuristieken}
Een metaheuristiek is een methode die een probleem optimaliseert door iteratief te proberen een kandidaatoplossing te verbeteren met betrekking tot een gegeven kwaliteitsmaat. 
Metaheuristieken zijn niet probleemspecifiek. 
Ze zijn in staat om grote ruimten met kandidaatoplossingen te doorzoeken, maar bieden geen garantie op het vinden van een optimale oplossing.

\TODO delta-evaluatie