\chapter{Heuristieken en metaheuristieken}

\section{Introductie}
Heuristieken zijn informele methoden die \emph{voldoende goede} oplossingen kunnen opleveren voor complexe problemen. 
Hierbij worden twee soorten onderscheiden: constructieve en perturbatieve heuristieken.
Constructieve heuristieken bouwen een oplossing op en de groep van perturbatieve heuristieken maken wijzigingen aan een al bestaande oplossing.

\section{Modelleringen}
Een modellering dient om een oplossing voor te stellen, waarbij deze aan de volgende voorwaarden moet voldoen:

\begin{itemize}
    \item Zo eenvoudig mogelijk
    \item Snel en eenvoudig te manipuleren
    \item Snel een eenvoudig te evalueren
\end{itemize}

\section{Lokale zoekmethoden}

\section{Metaheuristieken}